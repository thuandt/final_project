\chapter{Các công cụ tự động hóa hệ thống}

\newpage
\clearpage

Sự gia tăng của ảo hóa cùng với sức mạnh ngày càng tăng của các máy chủ chuẩn công nghiệp và sự sẵn có của điện toán đám mây đã dẫn đến một sự gia tăng đáng kể số lượng các máy chủ cần phải được quản lý trong và ngoài các doanh nghiệp, tổ chức. Trước kia, chúng ta có thể làm việc với hàng loạt các máy chủ vật lý hay truy cập trực tiếp vào các trung tâm dữ liệu nhưng giờ đây, chúng ta phải quản lý nhiều hơn rất nhiều các máy chủ và chúng lại còn được đặt ở rất nhiều các trung tập dữ liệu trên toàn thế giới.

Những trung tâm dữ liệu - đó chính là nơi các công cụ tổ chức và quản lý cấu hình vào cuộc. Trong nhiều trường hợp, chúng ta quản lý nhiều nhóm máy chủ giống hệt nhau, chạy các ứng dụng và dịch vụ giống hệt nhau. Chúng được triển khai trên các nền tảng ảo hóa trong các doanh nghiệp, tổ chức, hay ở trên hệ thống điện toán đám mây hoặc một máy chủ nào đó tại một trung tâm dữ liệu ở xa. Điều đó dẫn đến cần có một công cụ tự động hóa để có thể quản lý các cơ sở hạ tầng lớn và đang phát triển không ngừng này.

Puppet, Chef hay Ansible đã được xây dựng với suy nghĩ: đó là làm sao để có thể cấu hình và duy trì hàng chục, hàng trăm, thậm chí hàng ngàn máy chủ một cách dễ dàng và thuận tiện. Nói như vậy không có nghĩa rằng các doanh nghiệp - tổ chức nhỏ hơn sẽ không được hưởng lợi từ những công cụ này. Nói chung, các công cụ tự động hóa và sắp xếp công việc đều làm cho công việc dễ dàng hơn cho dù cơ sở hạ tầng lớn hay nhỏ.

Trong chương này chúng ta sẽ tìm hiểu thiết kế và chức năng của các công cụ tự động hóa hệ thống. Từ đó, đưa ra quyết định sử dụng công cụ cho phù hợp nhằm giải quyết bài toán trong thực tế mà chúng ta gặp phải.

\newpage
\clearpage
\section{Puppet}
Puppet là một framework mã nguồn mở và là công cụ để quản lý cấu hình của hệ thống máy tính. Trong phần này, chúng ta sẽ tìm hiểu tổng quan về Puppet: cách thức nó hoạt động, cách nó quản lý cấu hình của hệ thống máy chủ, lý do puppet có thể chạy trên nhiều nền tảng khác nhau. Tiếp đó, chúng ta sẽ tìm hiểu kiến trúc của Puppet, cách Puppet thu thập và quản lý các gói dữ liệu cấu hình.

\subsection{Tổng quan}
Puppet là một công cụ quản lý cấu hình mã nguồn mở viết bằng Ruby, được sử dụng để quản lý cấu hình máy chủ và tự động hóa hệ thống trong các trung tâm dữ liệu của Google, Twitter, thị trường chứng khoán New York, và nhiều doanh nghiệp lớn khác. Puppet được phát triển đầu tiên bởi Puppet Labs, và hiện tại Puppet Labs cũng là người duy trì chính của dự án này. Puppet được dùng để quản lý có khi chỉ vài máy chủ nhưng cũng có khi lên tới 50.000 máy chủ, cùng với đó là đội quản trị hệ thống từ một người tới hàng trăm người.

Puppet là một công cụ để quản lý cấu hình và bảo trì hệ thống máy tính; ngôn ngữ cấu hình của nó rất đơn giản. Chúng ta chỉ cần chỉ cho Puppet thấy chúng ta muốn cấu hình máy tính của chúng ta như thế nào, nó sẽ thực hiện đúng những gì chúng ta muốn. Khi hệ thống có sự thay đổi: chẳng hạn như một phiên bản cập nhật của gói phần mềm, thêm người dùng mới hay một cấu hình nào đó thay đổi, Puppet sẽ tự động cập nhật tất cả các máy chủ trong hệ thống đúng như cấu hình chúng ta muốn.

\newpage
\clearpage

\begin{lstlisting}[label={lst:puppet_ssh_pp},caption={Ví dụ về ngôn ngữ cấu hình của Puppet},morekeywords={class, package, file, source, ensure, require, service, File, Package}]
class ssh {
    package { ssh:
        ensure => installed
    }
    file { "/etc/ssh/sshd_config":
        source => 'puppet:///modules/ssh/sshd_config',
        ensure => present,
        require => Package[ssh]
    }
    service { sshd:
        ensure => running,
        require => [File["/etc/ssh/sshd_config"], Package[ssh]]
    }
}
\end{lstlisting}


Trong mã nguồn \ref{lst:puppet_ssh_pp} phía trên, Puppet đảm bảo rằng gói phần mềm \textbf{\textit{ssh}} được cài đặt và file cấu hình dịch vụ SSH \footnote{Secure Shell: \url{https://en.wikipedia.org/wiki/Secure_Shell}} \textbf{\textit{/etc/ssh/sshd\_config}} của tất cả các máy chủ trong hệ thống mà Puppet quản lý đều có cùng nội dung mới file đã định trước; thêm vào đó, Puppet đảm bảo việc dịch vụ này luôn được chạy giúp người quản trị hệ thống có thể can thiệp thủ công khi cần thiết.

\subsection{Kiến trúc hệ thống}

Puppet được xây dựng với hai chế độ làm việc:

\begin{itemize}
\item \textbf{Chế độ client/server} Có một máy chủ trung tâm với một dịch vụ chạy nền kết nối đến các "agents" chạy độc lập trên các máy trạm.

\item \textbf{Chế độ serverless} Chỉ có một tiến trình duy nhất thực hiện tất cả các công việc.
\end{itemize}

Để đảm bảo tính nhất quán giữa các chế độ, Puppet luôn có sự minh bạch trong các liên kết nội bộ của bản thân nó. Do đó, hai chế độ này sử dụng cùng một đường dẫn như nhau cho dù chúng có giao tiếp với nhau qua mạng hay không. Mỗi lệnh được thực thi cho dù lấy cấu hình từ bản nó hay một máy khác ở xa trong mạng thì chúng đều có một cách thực hiện như nhau. Tuy nhiên cũng phải lưu ý rằng, chế độ serverless là một phần trong trong mô hình client/server: tất cả các file cấu hình sẽ được đẩy xuống cho agents ở máy trạm xử lý, tại đây máy trạm chạy ở chế độ serverless sẽ làm việc trực tiếp với các tệp tin cấu hình và thực thi chúng. Phần này sẽ chỉ tập trung vào chế độ client/server bởi vì nó dễ hiểu hơn với các thành phân riêng biệt, nhưng hãy luôn nhớ rằng tất cả đều chạy ở chế độ serverless.

Một trong những lựa chọn ngay từ đầu trong kiến trúc của Puppet là máy trạm không nên truy cập trực tiếp (raw access) vào các module, thay vào đó, chúng lấy các cấu hình đã được chuẩn bị sẵn từ trước. Việc này cung cấp nhiều lợi ích:

\begin{itemize}
\item \textbf{Thứ nhất}, chung ta thực hiện được việc tối thiểu quyền hạn cần thiết. Trong đó mỗi máy chủ chỉ biết chính xác những gì nó cần phải biết (nó nên được cấu hình như thế nào), nhưng nó không biết (và không quan tâm) những máy trạm khác được cấu hình như thế nào.

\item \textbf{Thứ hai}, chúng ta hoàn toàn có thể phân tách các quyền cần thiết để tạo ra một cấu hình (bao gồm cả quyền truy cập vào nơi lưu trữ dữ liệu trung tâm) mà nó sẽ được thực hiện dưới máy trạm.

\item \textbf{Thứ ba}, chúng ta có thể chạy các máy trạm trong chế độ ngắt kết nối với máy chủ trung tâm, nhưng các cấu hình đã có của Puppet sẽ vẫn luôn được áp dụng. Nghĩa là, cho dù máy chủ trung tâm (puppet-master) không còn hoạt động hoặc không có kết nối đến nó thì mỗi máy trạm vẫn có thể làm việc độc lập.

\end{itemize}

\begin{figure}[h!]
    \begin{center}
    \fbox{\includegraphics[width=0.9\textwidth]{images/puppet_dataflow.png}}
    \end{center}
    \caption{Biểu đồ luồng dữ liệu của Puppet}
    \label{fig:puppet_dataflow}
\end{figure}

Với sự lựa chọn này, quy trình làm việc trở nên tương đối đơn giản

\begin{itemize}
\item Tiến trình trên máy trạm (Puppet agent) thu thập các thông tin về hệ thống mà nó đang làm việc, sau đó chuyển các thông tin này tới máy chủ trung tâm (Puppet Master)

\item Tại máy chủ trung tâm, các thông tin đó cùng các module trên ổ đĩa cục bộ được biên dịch thành một cấu hình cho một máy chủ cụ thể và trả lại nó cho các tiến trình trên máy trạm.

\item Các tiến trình trên máy trạm áp dụng những cấu hình cục bộ này và nó chỉ ảnh hưởng tới riêng máy trạm đó. Sau đó, các tập tin báo cáo được tạo ra rồi đưa kết quả về máy chủ trung tâm.
\end{itemize}

Vì thế, các agent có quyền truy cập vào thông tin riêng trên hệ thống của mình, các cấu hình của bản thân nó, cũng như các báo cáo nó tạo ra. Máy chủ trung tâm có bản sao của tất cả các dữ liệu này, cùng với quyền truy cập toàn bộ các module, cũng như bất kỳ cơ sở dữ liệu và dịch vụ nào khác dùng để biên dịch các cấu hình cần thiết.

\begin{figure}[h!]
    \begin{center}
    \fbox{\includegraphics[width=0.9\textwidth]{images/puppet_timing_diagram.png}}
    \end{center}
    \caption{Luồng dữ liệu lưu chuyển \\ giữa các thành phần và tiến trình của Puppet}
    \label{fig:puppet_timing_diagram}
\end{figure}

Ngoài các thành phần ở trong quy trình làm việc này, có rất nhiều loại dữ liệu được Puppet sử dụng cho các giao tiếp nội bộ của nó. Các loại dữ liệu rất quan trọng, bởi vì chúng hoàn thực hiện tất cả các thông tin liên lạc, đồng thời chúng cũng cung cấp các giao diện công cộng cho những công cụ khác sử dụng hay làm việc với chúng.

Các kiểu dữ liệu quan trọng nhất trong Puppet là:

\begin{itemize}
\item \textbf{Facts}: Thu thập các thông tin hệ thống trên mỗi máy trạm. Những thông tin này được dùng để biên dịch ra các cấu hình.

\item \textbf{Manifest}: Các tập tin chứa ngôn ngữ cấu hình Puppet, chúng thường được tổ chức thành các bộ sưu tập được gọi là module.

\item \textbf{Catalog}: Một đồ thị về các tài nguyên của máy chủ được quản lý và các rằng buộc giữa chúng.

\item \textbf{Report}: Tập hợp tất cả các sự kiện được tạo ra trong suốt quá trình tạo ra các Catalog.
\end{itemize}

Ngoài Facts, Manifests, Catalogs, and Reports, Puppet còn hỗ trợ các kiểu dữ liệu như các tệp tin, các chứng chỉ (được dùng trong việc xác thực) cùng nhiều kiểu dữ liệu khác.

\begin{figure}[h!]
    \begin{center}
    \fbox{\includegraphics[scale=1.0]{images/puppet_manifest_to_defined_state_unified.png}}
    \end{center}
    \caption{Cách Puppet biên dịch và thực hiện một manifest\\ trong chế độ Serverless}
    \label{fig:puppet_manifest_to_defined_state_unified}
\end{figure}

\begin{figure}[h!]
    \begin{center}
    \fbox{\includegraphics[width=0.85\textwidth]{images/puppet_manifest_to_defined_state_split.png}}
    \end{center}
    \caption{Cách Puppet biên dịch và thực hiện một manifest\\ trong chế độ Client-Server}
    \label{fig:puppet_manifest_to_defined_state_split}
\end{figure}

\clearpage
\subsection{Các thành phần chính}
Các thành phần chính của Puppet bao gồm: Agent, Facter, ENC, Compiler, Transaction, RAL và Reporting


\subsubsection{Agents}

Thành phần đầu tiên mà chúng ta tiếp xúc khi sử dụng Puppet là tiến trình agent. Trong các phiên bản trước của puppet, tiến trình này được tách riêng thành một tiến trình riêng biệt có tên là \textbf{\textit{puppetd}}, nhưng trong phiên bản 2.6 trở đi, chúng được tối giản hóa và bây giờ chúng được gọi bằng lệnh \textbf{\textit{puppet agent}}, tương tự như cách làm của Git\footnote{Một hệ thống quản lý mã nguồn phân tán \url{http://git-scm.com/}}. Các agent thực hiện rất ít các chức năng của riêng nó mà chủ yếu các công việc liên quan đến các cấu hình hay mã nguồn được thực hiện ở phía máy trạm như trong quy trình làm việc đã mô tả ở trên.


\subsubsection{Facter}

Thành phần tiếp theo sau agent là một công cụ bên ngoài gọi là \textbf{\textbf{facter}}, đó là một công cụ đơn giản được sử dụng để tìm kiếm và thu thập các thông tin về hệ thống nó đang chạy trên đó. Thông tin đó thường là hệ điều hành, địa chỉ IP, tên máy chủ, nhưng vì Facter là một công cụ có khả năng mở rộng nên rất nhiều các tổ chức thường thêm vào các plugin riêng họ để thu thập các thông tin khác mà họ cần. Các agent gửi những thông tin mà Facter đã thu thập được tới máy chủ trung tâm, nơi nó được tiếp quản và xử lý theo quy trình ở hình \ref{fig:puppet_dataflow}

\newpage
\clearpage

\subsubsection{External Node Classifier}

Trên máy chủ trung tâm, thành phần đầu tiên chúng ta cần đề cập tới là External Node Classifier\footnote{Các lớp phân loại mở rộng} hay ENC. ENC chấp nhận tên máy trạm hoặc trả về các cấu trúc dữ liệu đơn giản có chứa các cấu hình cấp cao của máy chủ đó. ENC nói chung là một ứng dụng hay dịch vụ riêng biệt: điển hình là một vài sản phẩm mã nguồn mở như Puppet Dashboard hay Foreman; đôi khi nó được tích hợp vào hệ thống dữ liệu có sẵn chẳng hạn như máy chủ LDAP... Mục đích cơ bản của ENC là để xác định những gì mà các lớp chức năng thuộc về, kèm theo đó là những thông số cấu hình cho các lớp này. Ví dụ, một máy trạm có thể thuộc lớp \textbf{\textit{debian}} hay \textbf{\textit{webserver}} và chúng có tham số để báo rằng chúng ở trung tâm dữ liệu tại \textbf{\textit{atlanta}}

Lưu ý rằng như Puppet 2.7, ENC không phải là một thành phần bắt buộc, thay vào đó người dùng có thể trực tiếp chỉ định cấu hình tại mỗi nút của Puppet. ENC được hỗ trợ trong khoảng 2 năm sau khi Puppet ra đời, bởi vì những nhà phát triển nhận ra rằng các lớp về cơ bản là khác biệt với các việc cấu hình chúng. Và, nó sẽ có ý nghĩa hơn nhiều khi đưa những vấn đề này vào các công cụ riêng biệt thay vì mở rộng ngôn ngữ để hỗ trợ cả hai cơ sở này. ENC luôn luôn khuyến khích sử dụng, và tại một số nơi nó trở thành một thành phần cần thiết (lúc này Puppet sẽ xuất hiện với tư cách một công cụ hữu ích hơn là một gánh nặng).

Một khi máy chủ trung tâm nhận được thông tin phân loại theo lớp từ ENC và thông tin hệ thống từ facter (thông qua các agent), nó đóng gói tất cả các thông tin vào một đối tượng nút (node object) và chuyển nó vào các trình biên dịch (Compiler).

\newpage
\clearpage

\subsubsection{Compiler}

Như đã đề cập ở trên, Puppet có một ngôn ngữ tùy chỉnh được xây dựng dành riêng cho việc cấu hình hệ thống. Trình biên dịch này bao gồm 3 khối: một bộ phân tích ngôn kiểu Yacc\footnote{https://en.wikipedia.org/wiki/Yacc} đã được tùy chỉnh; một nhóm các lớp được sử dụng để tạo ra Cây cú pháp trừu tượng (Abstract Syntax Tree hay AST); sau cùng là lớp biên dịch, nó xử lý các tương tác của tất cả các lớp, đồng thời cũng có chức năng như API và là một phần của hệ thống.


\subsubsection{Transaction}

Khi các Catalog đã được xây dựng thành công, nó được chuyển qua cho lớp Transaction. Trong một hệ thống mà máy trạm và máy chủ trung tâm được tách biệt, bộ phận này được chạy trên máy trạm, nó có nhiệm vụ kéo xuống các Catalog qua giao thức HTTP như Hình \ref{fig:puppet_timing_diagram}.

Lớp Transaction thực hiện một công việc tương đối đơn giản: tìm sự rằng buộc trong các mối quan hệ và đảm bảo các tài nguyên đấy được đồng bộ. Nó thực hiện một quá trình với 3 bước đơn giản: lấy trạng thái hiện tại của tài nguyên đó, so sánh nó với trạng thái mong muốn và thực hiện bất kì thay đổi cần thiết nào để sửa chữa sai lệnh.


\subsubsection{Resource Abstraction Layer}

Như đã biết, lớp Transcaction là rất quan trọng sự hoàn thành các công việc trong Puppet, nhưng thật ra, tất cả các công việc được thực sự thực hiện bởi lớp tài nguyên trừu tượng (Resource Abstraction Layer, viết tắt là RAL). Đây cũng là một trong các thành phần thú vị nhất của Puppet.

RAL là thành phần đầu tiên được tạo ra trong Puppet, hơn cả ngôn ngữ cấu hình, nó định nghĩa rõ ràng những gì con người có thể làm với nó. Công việc của RAL là xác định những gì là tài nguyên và cách các tài nguyên đó có thể thực hiện công việc của nó trên hệ thống. Sau đó ngôn ngữ Puppet được cụ thể hóa thành các mô hình bằng RAL. Bởi vì điều này nên nó là thành phần quan trọng nhất trong hệ thống, và khó khăn nhất để thay đổi. Có rất nhiều điều những nhà phát triển muốn khắc phục trong RAL, và họ đã thực hiện rất nhiều cải tiến quan trọng đối với nó trong nhiều năm qua (quan trọng nhất là việc bổ sung các Providers), nhưng vẫn có rất nhiều việc phải làm với RAL trong dài hạn.

Như đã nói, lớp Transaction không thực sự ảnh hưởng  trực tiếp đến hệ thống, mà nó dựa vào RAL để làm việc đó. Trong thực tế, providers là thành phần duy nhất của Puppet mà thực sự tác động vào hệ thống. Nếu lớp Transaction yêu cầu nội dung của một tập tin thì lớp Providers tìm kiếm và thu thập nó; nếu lớp Transaction xác định rằng nội dung của một tập tin nên được thay đổi thì lớp Providers sẽ thay đổi nó. Tuy nhiên phải lưu ý rằng, bản thân lớp Providers không bao giờ có quyền quyết định ảnh hưởng đến hệ thống mà chính lớp Transaction mới có quyền này, lớp Providers chỉ là người thực hiện công việc. Điều này cho phép các lớp Transaction kiểm soát hoàn toàn mà không yêu cầu phải hiểu bất cứ điều gì về các tập tin, người dùng, hoặc các gói. Sự phân chia rõ ràng này cho phép Puppet có sự mô phỏng đầy đủ nơi mà chúng ta có thể đảm bảo phần lớn hệ thống không bị ảnh hưởng.

\subsubsection{Reporting}

Trong quá trình lớp transaction đi theo các đồ thị và sử dụng RAL để thay đổi các cấu hình của hệ thống, các bản báo cáo được tạo ra. Báo cáo này bao gồm hầu hết cac sự kiện được tạo ra bởi những thay đổi trong hệ thống. Những sự kiện này, chúng là phản ảnh toàn diện về những công việc đã thực hiện: mốc thời gian thay đổi, giá trị trước đó, giá trị mới và bất kì thông báo nào được tạo ra, và tất nhiên là việc thay đổi đó thành công hay thất bại.

Các sự kiện được bao bọc trong một đối tượng ResourceStatus được ánh xạ vào từng tài nguyên. Vì vậy, đối với một giao dịch nhất định, chúng ta biết tất cả các nguồn tài nguyên đã được sử dụng, tất cả các thay đổi đã xảy ra, cùng với tất cả các metadata mà chúng ta cần biết về thay đổi đó.

Sau khi giao dịch hoàn tất, một số số liệu cơ bản được tính toán và lưu trữ trong báo cáo, và sau đó nó được gửi đến máy chủ trung tâm (nếu cấu hình). Với báo cáo đã gửi, quá trình cấu hình hoàn tất, các agent sẽ quay về trạng thái nghỉ hoặc là tự động kết thúc.


\subsection{Kiến trúc hạ tầng}

Trong mục này chúng ta sẽ tìm hiểu các thành phần hạ tầng của Puppet

\subsubsection{Hệ thống các Plugin}
Một trong những điều tuyệt vời của Puppet là nó có khả năng mở rộng rất tốt. Có ít nhất 12 loại mở rộng khác nhau của Puppet, có nghĩa nó có thể sư dụng cho bất kì ai. Ví dụ, chúng ta có thể bổ sung các tùy chỉnh trong các lĩnh vực sau:

\begin{itemize}
\item Các kiểu tài nguyên hay provider tùy chỉnh.
\item Cách xử lý các báo cáo cũng như việc lưu trữ các báo cáo này vào cơ sở dữ liệu riêng.
\item Bổ sung thêm các Indirector để tương tác với những cơ sở dữ liệu sẵn có.
\item Các facter để thu thập và cung cấp thêm các thông tin về máy trạm.
\end{itemize}

Tuy nhiên, do tính chất phân tán tự nhiên của Puppet nên cần có một nơi để các agent có thể tải về các plugin này. Vì vậy, mỗi lần chạy Puppet, điều đầu tiên chúng ta cần làm là tải về tất cả các plugin và đặt chúng ở máy chủ trung tâm. Các plugin này bao gồm các loại tài nguyên mới, các thông tin mới cần thu thập, hoặc một cách xử lý báo cáo khác thường nào đó.

Điều này khiến cho việc nâng cấp các agent của Puppet mà không bao giờ ảnh hưởng tới những gói cốt lõi nhất. Điều này vô cùng quan trọng trong các hệ thống Puppet có độ tùy chỉnh cao.

\subsubsection{Indirector}

Indirector là một dạng của chuẩn Inversion of Control\footnote{\url{https://en.wikipedia.org/wiki/Inversion_of_control}} (IoC) framework với khả năng mở rộng cao. IoC cho phép chúng ta tách biệt việc phát triển các chức năng với việc kiểm soát các chức năng mà bạn đang sử dụng. Trong trường hợp của Puppet, chúng cho phép nhiều plugin cùng cung cấp các chức năng chẳng hạn như việc trình biên dịch đưa thông tin qua HTTP hoặc tải nó khi nó đang chạy hay việc chuyển đổi giữa việc thay đổi cấu hình nhỏ hơn là thay đổi cả source code. Indirector là một dạng thể hiện đơn giản của IoC. Tất các lớp bắt tay nhau làm việc từ lớp này tới lớp khác thông qua Indirector bằng một giao diện chuẩn REST\footnote{\url{https://en.wikipedia.org/wiki/REST}}. Việc này có nghĩa có thể chuyển các máy trạm chạy ở chế độ Serverless với đầu cuối HTTP để lấy cái danh mục về thay vì sử dụng một đầu cuối đã được biên dịch.

\subsubsection{Hệ thống mạng}

Một câu hỏi được đặt ra khi viết nguyên mẫu của Puppet năm 2004 là sẽ sử dụng XMLRPC hay SOAP, những nhà phát triển đã chọn XMLRPC. Tất nhiên là chúng vẫn làm việc được nhưng nó gặp những vấn đề mà tất cả những người khác đều gặp phải: Nó không có một chuẩn giao diện giữa các thành phần, nó có xu hướng phức tạp rất nhanh.

Trong phiên bản 0.25 phát hành năm 2008, những nhà phát triển đã bắt đầu quá trình chuyển đổi tất cả các kết nối mạng sang dạng REST, nhưng họ đã chọn con đường phức tạp hơn là chỉ thay đổi kết nối mạng. Họ đã phát triển Indirector như một bộ khung tiêu chuẩn trong việc giao tiếp giữa các thành phần của Puppet, tuy nhiên lúc này REST chỉ là một lựa chọn trong cài đặt. Phải mất tới 2 phiên bản họ mới có bản hỗ trợ đầy đủ REST, tuy vậy họ vẫn chưa thực sự hoàn thành việc chuyển đổi để sử dụng JSON thay vì YAML. Họ chuyển sang JSON vì 2 lý do chính: xử lý YAML với Ruby chậm hơn rất nhiều lần so với việc xử lý JSON; đồng thời hầu hết các website đã chuyển sang dùng JSON, việc sử dụng JSON có vẻ đơn giản hơn nhiều so với YAML.

Các phiên bản mới của Puppet đã dần loại bỏ hoàn toàn các XMLRPC

\newpage
\clearpage

\subsection{Cài đặt và sử dụng}

Trong các phần trước chúng ta đã tìm hiểu cách thức hoạt động, kiến trúc hệ thống và hạ tầng của Puppet. Trong phần này, chúng ta sẽ tìm hiểu cơ bản về việc cài đặt và sử dụng Puppet trong môi trường Linux.

\subsubsection{Cách cài đặt Puppet trên một số HĐH Linux phổ dụng}
Puppet có thể được cài đặt và sử dụng trên rất nhiều các nền tảng khác nhau bao gồm:

\begin{itemize}
\item Red Hat Enterprise Linux (RHEL), CentOS, Fedora \& Oracle Enterprise Linux
\item Debian và Ubuntu
\item Mandrake và Mandriva
\item Gentoo
\item Solaris và OpenSolaris
\item MacOS X và MacOS X Server
\item *BSD - Các hệ điều hành họ BSD
\item AIX
\item HP UX
\item Microsoft Windows (tuy nhiên có một số hạn chế nhất định)
\end{itemize}

Trên một số nền tảng, Puppet có thể quản lý rất nhiều các loại cấu hình khác nhau bao gồm:
\begin{itemize}
  \item Files
  \item Services
  \item Packages
  \item Users
  \item Groups
  \item Cron jobs
  \item SSH keys
  \item Nagios configuration
\end{itemize}

Đối với Puppet, các agent và master server có cách cài đặt giống nhau, mặc dù hầu hết các hệ điều hành và bản phân phối các gói được phân chia theo chức năng của master và agent thành các gói riêng biệt. Trên một số hệ điều hành và bản phân phối, bạn cũng sẽ cần phải cài đặt Ruby cùng các thư viện của nó và số gói bổ sung cần thiết khác. Hầu hết các hệ thống đóng gói tốt sẽ có hầu hết các gói cần thiết, điển hình là Ruby - đó là yếu tố tiên quyết để có thể chạy được Puppet.

Trong nội dung hạn chế của đồ án này, chúng ta sẽ chỉ tìm hiểu cách cài đặt trên 2 bản phân phối Linux phổ biến nhất là RHEL và Debian (tương tự với CentOS hay Ubuntu). Chúng ta sử dụng repo của \textbf{Puppet Labs} để cài đặt Puppet vì ở đây là repo của nơi phát triển Puppet nên nó luôn đi kèm phiên bản mới cùng các bản vá bảo mật đi kèm.

\textbf{• Cách cài đặt Puppet trên RHEL}

Đối với máy chủ trung tâm:

\begin{lstlisting}[label={lst:puppet_install_master_rhel},caption={Cách cài đặt puppet master trên RHEL}, language=bash, deletekeywords={enable, true}]
# Install Puppet Labs repo
rpm -ivh http://yum.puppetlabs.com/puppetlabs-release-el-6.noarch.rpm
# Download puppet-server from Puppet Labs
yum install -y puppet-server
# Start puppet master
/etc/init.d/puppetmaster start
# Set Puppet Master to run on startup
puppet resource service puppetmaster ensure=running enable=true
\end{lstlisting}

Đối với agent:

\begin{lstlisting}[label={lst:puppet_install_agent_rhel},caption={Cách cài đặt puppet agent trên RHEL}, language=bash, deletekeywords={enable, true}]
# Add the puppet labs repo
rpm -ivh http://yum.puppetlabs.com/puppetlabs-release-el-6.noarch.rpm
# Install the puppet agent
yum install -y puppet
# Set Puppet Agent to run on startup
puppet resource service puppet ensure=running enable=true
\end{lstlisting}

\textbf{• Cách cài đặt Puppet trên Debian}

Đối với máy chủ trung tâm:

\begin{lstlisting}[label={lst:puppet_install_master_debian},caption={Cách cài đặt puppet master trên Debian}, language=bash, deletekeywords={enable, true}]
# Download Puppet Labs repo
wget http://apt.puppetlabs.com/puppetlabs-release-wheezy.deb
# Install Puppet Labs repo
sudo dpkg -i puppetlabs-release-wheezy.deb
# Install puppetmaster package
sudo apt-get update
sudo apt-get install puppetmaster
# Set Puppet Master to run on startup
puppet resource service puppetmaster ensure=running enable=true
\end{lstlisting}

Đối với agent:

\begin{lstlisting}[label={lst:puppet_install_agent_debian},caption={Cách cài đặt puppet agent trên Debian}, language=bash, deletekeywords={enable, true}]
# Download Puppet Labs repo
wget http://apt.puppetlabs.com/puppetlabs-release-wheezy.deb
# Install Puppet Labs repo
sudo dpkg -i puppetlabs-release-wheezy.deb
# Install puppet package
sudo apt-get update
sudo apt-get install puppet
# Set Puppet Agent to run on startup
puppet resource service puppet ensure=running enable=true
\end{lstlisting}

\subsubsection{Cách cấu hình cơ bản hệ thống Puppet}
Như đã nói ở trên, tuy các nền tảng khác nhau có các cách cài đặt khác nhau nhưng việc cấu hình trên các nền tảng đều tương tự nhau.

Các agent liên lạc với master qua các kênh SSL mã hóa, vì thế, đầu tiên chúng ta phải thiết lập kênh liên lạc cho các agent tới master.

\begin{lstlisting}[label={lst:puppet_config_ssl},caption={Thiết lập kết nối qua kênh SSL}, language=bash, deletekeywords={test}]
# On the client machine, request a certificate to master:
puppet agent --server puppet --waitforcert 60 --test
# On the Puppetmaster machine, view a list of all certificates waiting to be signed:
puppet cert --list
#Sign the certificate for all clients
puppet cert --sign --all
\end{lstlisting}

\textbf{Chú ý}: Puppet master phải mở firewall ở port 8140 để các agent có thể kết nối tới được.

Sau khi đã thiết lập các kênh liên lạc từ các agent tới master, chúng ta bắt đầu quá trình thiết lập cấu hình cho các node. Sẽ có rất nhiều các thiết lập phải cấu hình tùy theo nhu cầu hệ thống thực tế. Ví dụ dưới dây là một thiết lập cho \textbf{class sudo}

\begin{lstlisting}[label={lst:puppet_config_class_sudo},caption={Cấu hình class sudo của Puppet},morekeywords={class, file, source, owner, group, mode, source}]
# /etc/puppet/manifests/classes/sudo.pp

class sudo {
  file { "/etc/sudoers":
    owner => "root",
    group => "root",
    mode  => 440,
    source => "puppet://puppet.labs/files/sudoers"
  }
}
\end{lstlisting}

Các thiết đặt cùng cấu hình chi tiết hơn chúng ta có thể tham khảo trên trang chủ tài liệu của Puppet tại địa chỉ: \url{http://docs.puppetlabs.com}. Tài liệu ở đây khá đầy đủ và chi tiết.

\subsection*{Tóm lại}

Puppet là một hệ thống các công cụ cả đơn giản lẫn phức tạp. Puppet là một framework điển hình để giải quyết các vấn đề liên quan đến cấu hình trong tự động hóa hệ thống. Tuy vậy nó là một ứng dụng đơn giản và dễ tiếp cận.

\newpage
\clearpage
\section{Chef}

\subsection{Tổng quan}
Chef là một framework cho việc tự động hóa hệ thống và cơ sở hạ tầng điện của toán đám mây. Chef được dùng để triển khai các máy chủ hoặc các ứng dụng tới bất kỳ đâu: từ máy chủ vật lý tới máy chủ ảo hay máy chủ trên hệ thống điện toán đám mây, mà không bao giờ phải bận tâm về kích thước của cơ sở hạ tầng. Chef-client dựa trên các định nghĩa trừu tượng (được gọi là cookbooks và recipes) được viết bằng Ruby và được quản lý như mã nguồn. Mỗi định nghĩa mô tả cách một phần cụ thể của cơ sở hạ tầng nên được xây dựng hoặc quản lý. Chef-client sau đó áp dụng những định nghĩa này cho các máy chủ và các ứng dụng, theo quy định, dẫn đến một cơ sở hạ tầng hoàn toàn tự động. Khi một node mới được đưa vào hệ thống, điều duy nhất mà chef-client cần biết là có những cookbooks và recipes nào cần phải được áp dụng.

\subsection{Kiến trúc hệ thống}

Sơ đồ dưới đây cho thấy mối quan hệ giữa các yếu tố khác nhau của Chef, bao gồm các nút, máy chủ, và các máy trạm. Những yếu tố này làm việc cùng nhau để cung cấp các chef-client các thông tin và hướng dẫn giúp nó thực hiện các công việc của mình.

\newpage
\clearpage

\begin{figure}[h!]
    \begin{center}
    \fbox{\includegraphics[width=\textwidth]{images/chef_overview.png}}
    \end{center}
    \caption{Mối quan hệ giữa các thành phần trong kiến trúc của Chef}
    \label{fig:chef_overview}
\end{figure}

Chef bao gồm ba yếu tố chính : một máy chủ trung tâm, một hay nhiều các nút, và ít nhất một máy trạm.

\begin{itemize}
\item Chef-server là máy chủ trung tâm. Nó phải luôn online để đảm bảo các chef-client có thể lấy được các cookbook và recipes để áp dụng.
\item Các máy trạm là nơi mà từ đó các cookbook và recipes được viết ra, các chính sách được định nghĩa. Dữ liệu ở đây được đồng bộ với chef-repo và cuối cùng tải lên server
\item Mỗi nút trong hệ thống đều có một chef-client quản lý. Chef-client có nhiệm vụ tự động hóa các công việc mà mỗi nút được yêu cầu.
\end{itemize}

Cookbook là một yếu tố rất quan trọng và có thể coi là một thành phần riêng biệt (cùng với máy chủ, máy trạm và các nút). Nhìn chung, cookbook được viết ra và quản lý bởi các máy trạm, sau đó được chuyển lên server; từ server chúng được đẩy xuống các nút bằng chef-client trong mỗi lần chef-client thực thi.

\subsection{Các thành phần chính}
Phần dưới đây sẽ cho chúng ta biết chi tiết hơn về các thành phần đã kể trên

\textbf{\large Các nút}
\addcontentsline{toc}{paragraph}{Các nút}


Một nút trong hệ thống của Chef có thể là một máy chủ vật lý, máy chủ ảo hoặc là một máy chủ trên mây, hoặc một thiết bị mạng. Chúng được cấu hình và duy trì bởi chef-client.

Một nút dựa trên đám mây thường được lưu trữ trong một số dịch vụ dựa trên điện toán đám mây, chẳng hạn như Amazon Virtual Private Cloud, OpenStack, Rackspace, Google Compute Engine, Linode, hoặc Windows Azure. Knife là công cụ của người quản trị hệ thống\footnote{Trong kiến trúc của chef thì knife có nghĩa là người quản trị hệ thống}, nó đã hỗ trợ sẵn các dịch vụ này. Knife có thể sử dụng các plugin để tạo ra các máy chủ trên hệ thống đám mây. Sau khi máy chủ đó được tạo ra, chef-client có thể được sử dụng để triển khai, cấu hình và duy trì những máy chủ này.

Một nút vật lý thường là một máy chủ hoặc một máy ảo, nhưng nó có thể là bất kỳ thiết bị gắn liền với hệ thống có khả năng gửi, nhận, và chuyển tiếp thông tin trên một kênh giao tiếp. Nói cách khác, một nút vật lý là bất kỳ thiết bị hoạt động nào gắn liền với hệ thống mà có thể chạy chef-client và cũng cho phép chef-client giao tiếp với máy chủ.

Các nút ảo thường mà các máy chủ ảo được tạo ra nhờ các phần mềm ảo hóa, cho dù vậy chúng có tính chất tương đương một nút vật lý.

Một nút mạng thường là các thiết bị mạng như swicth, router, VLAN hay những thứ tương tự mà có thể quản lý bằng chef-client.

Các thành phần quan trọng của nút bao gồm:

\textbf{Chef-client}

Chef-client là một agent chạy cục bộ trên tất cả các nút đã được đăng kí với máy chủ. Khi chef-client chạy, nó sẽ thực hiện tất cả các bước được yêu cầu để đưa nút vào trạng thái mong muốn. Những việc này bao gồm:

\begin{itemize}
\item Đăng ký và chứng thực các nút với máy chủ
\item Xây dựng các đối tượng nút
\item Đồng bộ hóa các cookbook
\item Biên dịch các tập hợp tài nguyên bằng cách tải về cookbook cần thiết, bao gồm cả recipes, các thuộc tính, cùng tất cả phụ thuộc khác
\item Thực hiện các hành động thích hợp và cần thiết để cấu hình các nút
\item Tìm kiếm ngoại lệ và các thông báo, và xử lý mỗi khi có yêu cầu
\end{itemize}

Cặp khóa công khai RSA được sử dụng để xác thực các chef-client với máy chủ mỗi khi chef-client cần truy cập vào dữ liệu được lưu trữ trên máy chủ. Điều này ngăn cản bất kỳ nút nào truy cập dữ liệu mà nó không nên truy cập và đồng thời cặp khóa này đảm bảo rằng chỉ có các nút đã được đăng ký với máy chủ có thể quản lý được các tài nguyên.

\textbf{Ohai}

Ohai là một công cụ được sử dụng để phát hiện các thuộc tính trên một nút, sau đó cung cấp những thuộc tính này cho chef-client tại mỗi lần chạy. Ohai là thành phần bắt buộc phải có của chef-client và nhất thiết phải có mặt trong một nút. Các loại thuộc tính mà Ohai thu thập bao gồm:

\begin{itemize}
\item Thông tin chi tiết về nền tảng
\item Thông tin về mức sử dụng tài nguyên mạng
\item Thông tin về mức sử dụng bộ nhớ
\item Thông tin về mức sử dụng CPU
\item Các thông tin về nhân hệ điều hành
\item Tên của máy chủ
\item Tên miền đầy đủ của máy chủ
\item Chi tiết cấu hình khác
\end{itemize}

Những thuộc tính được thu thập bởi Ohai là các thuộc tính tự động, trong đó các thuộc tính được sử dụng bởi các chef-client để đảm bảo rằng những thuộc tính này vẫn không thay đổi sau khi các chef-client được thực hiện cấu hình các nút.

\newpage
\clearpage

\textbf{\large Máy trạm}
\addcontentsline{toc}{paragraph}{Máy trạm}


Một máy trạm là một máy tính được cấu hình để chạy Knife. Nó cũng dùng để đồng bộ hóa với các chef-repo, cũng tương tác với một máy chủ duy nhất. Vị trí của máy trạm là nơi mà từ đó người dùng sẽ làm hầu hết công việc của họ, bao gồm:

\begin{itemize}
\item Phát triển các cookbook và recipes (sử dụng ngôn ngữ lập trình Ruby)
\item Đồng bộ hóa với chef-repo sử dụng hệ thống quản lý mã nguồn (Git hoặc SVN)
\item Sử dụng Knife để tải lên cái item từ chef-repo lên server
\item Cấu hình các chính sách của tổ chức, bao gồm cả việc định nghĩa các vai trò (roles) và môi trường, đồng thời đảm bảo việc dữ liệu quan trọng đã được lưu lại trong data bags
\item Tương tác với các nút khi cần thiết, chẳng hạn thiết lập bootstrap cho một nút.
\end{itemize}

\textbf{\large Knife}
\addcontentsline{toc}{paragraph}{Knife}


Knife là một công cụ dòng lệnh cung cấp một giao diện giữa các chef-repo cục bộ tại máy trạm và máy chủ. Knife giúp cho người dùng có thể quản lý những điều sau:

\begin{itemize}
\item Các nút
\item Cookbooks và recipes
\item Roles
\item Lưu trự dữ liệu JSON (data bags), bao gồm cả dữ liệu mã hóa
\item Các thiết lập môi trường
\item Cái tài nguyên của điện toán đám mây, bao gồm cả tài nguyên dự phòng
\item Cài đặt các chef-client trên máy trạm quản lý
\item Tìm kiếm các thông tin được index trên máy chủ
\end{itemize}

\textbf{\large Chef-Repo}
\addcontentsline{toc}{paragraph}{Chef-Repo}


Chef-repo là nơi mà các đối tượng dữ liệu sau được lưu trữ:

\begin{itemize}
\item Cookbooks (bao gồm cả recipes, phiên bản, các thuộc tính, tài nguyên, nhà cung cấp, thư viện và các mẫu)
\item Roles
\item Databags
\item Các thông số về môi trường làm việc
\item Các file cấu hình (cho máy khách, máy trạm và máy chủ)
\end{itemize}

Chef-repo nằm trên một máy trạm và cần được đồng bộ hóa với hệ thống quản lý mã nguồn, chẳng hạn như git. Tất cả các dữ liệu trong chef-repo nên được đối xử như là mã nguồn.

Knife được sử dụng để upload dữ liệu từ chef-repo lên máy chủ. Sau khi tải lên, dữ liệu được các chef-client sử dụng để quản lý tất cả các nút đã được đăng kí, đồng thời đảm bảo tất cả những thứ như cookbooks, cái thống số môi trường, roles và các thiết lập khác được áp dụng cho nút đó một cách đứng đắn.

\newpage
\clearpage

\textbf{\large Máy chủ}
\addcontentsline{toc}{paragraph}{Máy chủ}


Các máy chủ hoạt động như một trung tâm chứa dữ liệu cấu hình. Các máy chủ lưu trữ các cookbooks, chính sách sẽ được áp dụng cho các nút, và các metadata được quản lý bởi chef-client. Các nút sử dụng chef-client để yêu cầu máy chủ các thông tin chi tiết về cấu hình, chẳng hạn như recipes, templates hay các tệp tin. chef-client thực hiện các công việc của mình trên từng nút mà không phải là trên máy chủ. Cách tiếp cận theo hướng phân phối mở rộng này khá tương ứng với mô hình của Puppet.

Hosted Enterprise Chef là một phiên bản máy chủ của Chef. Hosted Enterprise Chef được xây dựng dựa trên điện toán đám mây, có khả năng mở rộng và luôn sẵn sàng 24x7/365. Hosted Enterprise Chef có thể quản lý bất kì máy chủ nào mà không yêu cầu nó phải được thiết lập và quản lý từ phía sau tường lửa.

\textbf{\large Cookbooks}
\addcontentsline{toc}{paragraph}{Cookbooks}


Cookbook là đơn vị cơ bản của Chef trong việc cấu hình và phân phối các chính sách. Mỗi cookbook định nghĩa một kịch bản, chẳng hạn như tất cả mọi thứ cần thiết để cài đặt và cấu hình MySQL, và sau đó nó chứa tất cả các thành phần được yêu cầu để hỗ trợ kịch bản đó, bao gồm:

\begin{itemize}
\item Các giá trị của thuộc tính được thiết lập trên các nút
\item Định nghĩa sự cho phép tạo ra hoặc sử dụng lại tập hợp các tài nguyên
\item Các thư viện dùng để mở rộng chef-client hoặc cung cấp sự trợ giúp bằng mã nguồn Ruby
\item Recipes là cách thức xác định các tài nguyên cần thiết để quản lý và thứ tự của chúng khi áp dụng
\item Các mẫu
\item Các phiên bản
\item Các siêu dữ liệu về cách thức (bao gồm cả các gói phụ thuộc), hạn chế phiên bản, nền tảng hỗ trợ hay những gì tương tự.
\end{itemize}

Chef-client sử dụng ngôn ngữ Ruby như một ngôn ngữ tham chiếu để tạo ra cookbook và xác định recipes, với DSL mở rộng cho các tài nguyên cụ thể. Chef-client cung cấp một tập hợp lý các tài nguyên, đủ để hỗ trợ rất nhiều các kịch bản tự động hóa cơ sở hạ tầng phổ biến nhấ. Tuy nhiên, DSL này cũng có thể được mở rộng khi có thêm tài nguyên và khả năng được yêu cầu.

\newpage
\clearpage

\subsection{Cài đặt và sử dụng}

\textbf{\large Cách cài đặt Chef (server và client)}
\addcontentsline{toc}{paragraph}{Cách cài đặt}

Hiện tại Chef-Server chỉ hỗ trợ gói cài đặt cho RHEL và Ubuntu. Chef-Client hỗ trợ đa nền tảng hơn hơn và có script cài đặt đi kèm.

\textbf{• Cách cài đặt Chef trên RHEL}

Đối với Chef-Server:

\begin{lstlisting}[label={lst:chef_install_server_rhel},caption={Cách cài đặt chef-server trên RHEL},language=bash]
# Install chef-server
rpm -ivh https://opscode-omnibus-packages.s3.amazonaws.com/el/6/x86_64/chef-server-11.0.10-1.el6.x86_64.rpm
# Initial configuration for chef-server
chef-server-ctl reconfigure
\end{lstlisting}

Đối với Chef-Client:

\begin{lstlisting}[label={lst:chef_install_client_rhel},caption={Cách cài đặt chef-client trên RHEL},language=bash]
# Install chef-client
curl -L https://www.opscode.com/chef/install.sh | bash
\end{lstlisting}

\textbf{• Cách cài đặt Chef trên Ubuntu}

Đối với Chef-Server:

\begin{lstlisting}[label={lst:chef_install_server_ubuntu},caption={Cách cài đặt chef-server trên Ubuntu},language=bash]
# Download chef-server package
wget https://opscode-omnibus-packages.s3.amazonaws.com/ubuntu/12.04/x86_64/chef-server_11.0.10-1.ubuntu.12.04_amd64.deb
# Install chef-server package
sudo dpkg -i chef-server_11.0.10-1.ubuntu.12.04_amd64.deb
# Initial configuration for chef-server
sudo chef-server-ctl reconfigure
\end{lstlisting}

\newpage
\clearpage

Đối với Chef-Client:

\begin{lstlisting}[label={lst:chef_install_client_ubuntu},caption={Cách cài đặt chef-client trên Ubuntu},language=bash]
# Install chef-client
curl -L https://www.opscode.com/chef/install.sh | bash
\end{lstlisting}

\textbf{Chú ý}: Chef-Server cần mở cổng 443 để các Chef-Client có thể liên lạc được với nó.

\textbf{• Cách cài đặt Knife trên máy trạm}

Như đã nói trong kiến trúc của Chef, cần có một máy trạm chứa công cụ Knife và chef-repo để đưa các lệnh tới máy chủ và các nút.

Vì vậy chúng ta phải cài đặt chef-client trên máy trạm này, cài đặt công cụ knife và đồng bộ chef-repo.

\begin{lstlisting}[label={lst:chef_install_knife},caption={Cách cài đặt knife và chef-repo},language=bash]
# Install chef-client
curl -O -L http://www.opscode.com/chef/install.sh | bash

# Clone the Chef Repo skeleton directory to work in:
cd ~/Development
git clone https://github.com/opscode/chef-repo.git
\end{lstlisting}

\newpage
\clearpage

\textbf{\large Cách cấu hình cơ bản hệ thống Chef}
\addcontentsline{toc}{paragraph}{Cách cấu hình cơ bản}

Các bước cấu hình của Chef phức tạp hơn so với Puppet khá nhiều, ở đây chúng ta chỉ giới thiệu một số bước cấu hình đơn giản. Chi tiết về các cấu hình còn lại có thể tham khảo trên trang tài liệu của dự án: 

\url{http://docs.opscode.com}

\begin{lstlisting}[label={lst:chef_basic_config},caption={Các bước cấu hình hệ thống Chef},language=bash]
# Config knife
knife configure
...

# List chef-client (nodes)
knife client list

# Bootstrap first client server
knife bootstrap -u $USERNAME --sudo $FQDN_OF_CLIENT_SERVER
\end{lstlisting}

\newpage
\clearpage

\subsection*{Tóm lại}

Các nguyên tắc cơ bản quan trọng của Chef là chúng ta (những người sử dụng) là những người biết rõ nhất về những gì môi trường của bản thân, những gì cần làm, và làm thế nào để duy trì được nó. Chef-client được thiết kế để không phá vỡ những điều như vậy. Và chỉ có chúng ta cũng như đồng nghiệp của chúng ta mới hiểu rõ ràng các vấn đề về kĩ thuật cũng như phải làm gì để giải quyết chúng. Chỉ chúng ta mới có thể hiểu được các vấn đề về con người(trình độ kỹ năng, những kĩ năng kiểm soát, và các vấn đề nội bộ khác), những thứ mà duy nhất chỉ doanh nghiệp hay tổ chức của bạn biết được giải pháp.

Ý tưởng ở đây là bạn nắm rõ trong tay những gì sẽ xảy ra với hệ thống của bạn, và cách như thế nào để làm nó hoạt động hiệu quả. Tuy nhiên, một cá nhân rất khó có thể nắm bắt được mọi thứ về một vấn đề phức tạp, cũng như các bước để giải quyết chúng. Vấn đề cũng tương tự với những công cụ này. Chef hỗ trợ chúng ta quản lý các cơ sở hạ tầng. Chef có thể giúp giải quyết các vấn đề phức tạp. Chef có một cộng đồng rộng lớn, nơi mọi người có rất nhiều các kinh nghiệm trong việc giải quyết các vấn đề phức tạp và họ có thể cung cấp các kiến thức hay hỗ trợ chúng trong lĩnh vực mà tổ chức chúng ta chưa có kinh nghiệm. Cộng đồng cũng với Chef có thể giúp doanh nghiệp hay tổ chức của chúng ta giải quyết các vấn đề phức tạp.

\newpage
\clearpage
\section{Ansible}
Ansible là một giải pháp đơn giản được sử dụng trong việc tự động hóa hàng loạt các công việc liên quan đến hạ tầng CNTT như tự động cấu hình, tự động triển khai phần mềm, và nhiều công việc khác nữa. Trong mô hình của Ansible, hạ tầng CNTT của bạn được nhìn góc độ là một kiến trúc tổng thể của các thành phần có liên quan thay vì chỉ quản lý một hệ thống tại một điểm riêng lẻ. Nó không sử dụng các agent hoặc thêm vào các lớp tùy chọn bảo mật bổ sung, do đó việc triển khai Ansible vô cùng đơn giản. Một điều quan trọng nữa đó là ngôn ngữ cấu hình mà nó sử dụng rất đơn giản (được gọi là playbooks), nó cho phép mô tả những công việc tự động bằng tiếng Anh đơn thuần thay vì viết một điều gì đó phức tạp bằng một ngôn ngữ lập trình nào đó. Bằng cách sử dụng Ansible, chúng ta sẽ thực hiện việc tự động hóa hàng loạt nhanh hơn, thậm chí nó có thể đạt tới những điều mà ta chưa từng thấy trước đó.


\subsection{Tổng quan}

Ansible là một phần mềm mã nguồn mở dùng trong việc quản lý cấu hình của hạ tầng CNTT, triển khai sản phẩm công nghệ, cùng như điều phối các hoạt động tự động hóa khác. Với chỉ một công cụ duy nhất, nó đem lại những hiệu quả lớn trước hàng loạt các thách thức về tự động hóa. Ansible cung cấp sự thay thế hiệu các chức năng cốt lỗi vốn có trong các giải pháp tự động hóa khác, đồng thời tìm kiếm lời giải cho những vấn đề chưa được giải quyết. Bao gồm sự phối hợp rõ ràng về các quy trình làm việc phức tạp và sự thống nhất về cấu hình của hệ điều hành và phần mềm ứng dụng triển khai.

Ansible tìm cách giữ cho những mô tả về các quy trình công việc dễ hiểu và có thể được thực hiện nhanh chóng. Điều đó đồng nghĩa với việc những người mới sử dụng Ansbile có thể nhanh chóng hòa nhập vào các dự án mới, đồng thời dễ dàng hiểu được những công việc cho dù họ có tham gia vào dự án muộn hơn. Không chỉ vậy, Ansible luôn tìm cách tạo ra những công cụ thật mạnh mẽ cho những chuyên gia, nhưng bình đẳng cho tất cả các cấp độ kỹ năng của người sử dụng. Từ đó, rút ngắn thời gian đưa sản phẩm ra thị trường; giảm thiểu các lỗi có thể xảy ra do sự thay đổi cấu hình của hạ tầng CNTT.

Ansible được thiết kế nhỏ gọn, tiện dụng, an toàn, và có độ tin cậy cao, không mất nhiều công sức trong việc học sử dụng cho dù là quản trị viên, nhà phát triển hay nhà quản lý.

\subsection{Kiến trúc hệ thống}

\begin{figure}[h!]
    \begin{center}
    \fbox{\includegraphics[width=\textwidth]{images/ansible_architect.jpg}}
    \end{center}
    \caption{Kiến trúc hệ thống của Ansible}
    \label{fig:ansible_arch}
\end{figure}

\newpage
\clearpage

Một trong những khác biệt chính giữa Ansible với những sản phẩm cùng loại chính là kiến trúc của nó

\begin{itemize}
\item Mặc định, Ansible quản lý các máy trạm thông qua giao thức SSH. Nó sử dụng một thư viện được gọi là \textbf{paramiko} (được viết bằng lập trình Python\footnote{\url{https://en.wikipedia.org/wiki/Python_(programming_language)}}) hoặc sử dụng ngay OpenSSH của hệ điều hành.

\item Ansible có thể truyền tải theo khác nhau, các phương thức là có thể thay đổi được. Ví dụ: Mặc dù \textbf{0mq} - một phương thức truyền tải tăng tốc (accelerated transport) được đưa ra nhưng Ansible hỗ trợ cả phương thức không sử dụng mạng.

\item Ansible không yêu cầu quyền root\footnote{account có quyền tuyệt đối với hệ thống trên Linux, BSD hay Solaris} để truy cập mà nó có thể cấu hình để dùng sudo\footnote{\url{https://en.wikipedia.org/wiki/Sudo}} trong các trường hợp cần thiết.

\item Ansible không yêu cầu một khóa SSH cụ thể hay một người dùng riêng. Nó có thể làm việc với bất cứ người sử dụng được cung cấp, nghĩa là Ansible tôn trọng quyền truy cập của hệ điều hành của bạn.

\item Khi có yêu cầu, Ansible sẽ chuyển các module cần thiết tới các nút điều khiển, sau đó chạy chúng từ xa với các thông tin người dùng đã được cung cấp và không để lại được bất cứ cài đặt gì trên các nút này.

\item Ansible không yêu cầu bất kỳ phần mềm máy chủ được chạy từ một máy quản lý, nó chỉ yêu cầu các thông tin đăng nhập của người dùng mà thôi.

\item Ansible không yêu cầu bất kì một phần mềm agent nào được chạy trên cái nút điều khiển.

\item Ansible không cần mở thêm bất cứ một cổng nào ngoài SSH cũng như không yêu cầu phải có hạ tầng PKI để bảo trì.

\item Những người có quyền truy cập vào máy chủ điều khiển (hoặc máy chủ điều khiển mã nguồn) không thể xóa hay thay đổi các nội dung các máy trạm (hoặc ra lệnh cho chúng chạy một lệnh nào đó) mà không có cũng có thông tin về hệ thống đó.

\item Khi không còn quản lý nữa, Ansible không dùng đến bất cứ tài nguyên nào của những máy trạm.

\end{itemize}

Những đặc điểm trên kết hợp với nhau làm cho Ansible trở nên lý tưởng cho các môi trường bảo mật hoặc hiệu suất cao, nơi có những lo ngại về sự ổn định hoặc việc thay đổi thường xuyên của các agent. Những thuộc tính trên hầu hết đều hữu ích trong các lĩnh vực về máy tính.

Ansible là sự thiết kế đồng nhất giữa kinh nghiệm của sử dụng và phương pháp tiếp cận. Ansible được thiết kế để làm cho việc cấu hình và xử lý các hạ tầng CNTT cũng chỉ đơn giản như việc đọc hoặc viết cấu hình, thậm chí đối với những người chưa qua đào tạo về việc đọc chúng.

Mặc dù Ansible có thể thực hiện hầu hết các loại nhiệm vụ tự động hóa, Ansible không giống với ngôn ngữ lập trình phần mềm, nó chỉ là các mô tả cơ bản về trạng và tiến trình. Hơn nữa, nó cố gắng để giải quyết nhiều vấn đề chồng chéo của tự động hóa hệ thống từ một framework duy nhất với mục tiêu giảm thiểu thời gian và chi phí để học và hiểu cũng như gắn kết nhiều framework với nhau.

Với các phương pháp truyền thống khác, người dùng thường phải kết hợp nhiều công cụ với nhau có để bao quát hết những điều cơ bản trong quản lý hệ thống CNTT và cấu hình phần mềm, bao gồm:

\begin{itemize}
\item Một công cụ dùng quản lý cấu hình, nó dùng để làm việc với các hệ điều hành cơ bản, mô tả các trạng thái mong muốn của một hệ thống, nhưng không phải quá trình để đưa nó vào trạng thái đó.

\item Một công cụ triển khai, dùng để đưa sản phẩm phần mềm lên hệ thống, nó tập trung vào quá trình thực hiện.

\item Một công cụ dùng để thực thi, cho các tác vụ thực thi tức thời - những thứ mà không phù hợp với mô hình trước đây, chẳng hạn như restart hàng loạt các máy chủ.
\end{itemize}

Ansible đã gộp tất cả các yếu tó đó thành một công cụ duy nhất, đồng thời cung cấp các khả năng và đặc điểm cho phép thực hiện triển khai những phần mềm nhiều lớp và các quy trình làm việc phức tạp.

\subsection{Các thành phần chính}

\newpage
\clearpage

\section{Tổng kết chương}

Với những đặc điểm và kiến trúc đã được nêu trong các mục ở trên, dễ dàng nhận thấy, Puppet hay Chef sẽ hấp dẫn các nhà phát triển hay những hệ thống phát triển có định hướng, Ansible được dùng nhiều hơn cho các nhu cầu của quản trị hệ thống. Giao diện đơn giản và tính hữu dụng của Ansbile rất phù hợp những suy nghĩ của người quản trị hệ thống. Và trong một hệ thống với rất nhiều các hệ điều hành họ Linux và Unix, việc triển khai hệ thống sử dụng Ansible trở trên rất dễ dàng và nhanh chóng.

Puppet là framework trưởng thành nhất và cũng gần gũi nhất từ quan điểm khả năng sử dụng, mặc dù Puppet yêu cầu phải có nền tảng kiến thức vững chắc về ngôn ngữ lập trình Ruby. Puppet không ở dạng streamlined như Ansible. Puppet là phương án an toàn nhất đối với các môi trường không đồng nhất, tuy vậy Ansbile cũng có thể tốt và phù hợp hơn trong một hệ thống có cơ sở hạ tầng lớn hơn hoặc phức tạp hơn.

Chef khá ổn định và được thiết kế tốt nhưng những tính năng của nó vẫn chưa đạt đến độ chín như của Puppet. Chef có thể gây ra một số khó khăn trong việc sử dụng nó đối với các quản trị hệ thống có ít kinh nghiệm về lập trình, tuy vậy nó cũng có thể phù hợp một cách tự nhiên cho những quản trị hệ thống có đầu óc lập trình.

\newpage
\clearpage

\pagestyle{empty}
\begin{landscape}
\begin{longtable}{|l|*3{p{6.5cm}|}}
  \caption {Bảng so sánh ưu nhược điểm của các công cụ tự động hóa} \\
  \hline \multicolumn{1}{|l|}{~} & \multicolumn{1}{|l|}{\textbf{Puppet}} & \multicolumn{1}{|l|}{\textbf{Chef}} & \multicolumn{1}{|l|}{\textbf{Ansible}} \\ \hline
  \endfirsthead
  \textbf{Ưu điểm}
  & • Các module được viết bằng ngôn ngữ Ruby
  & • Cookbooks và recipes có thể tận dụng toàn bộ sức mạnh của ngôn ngữ Ruby
  & • Các module có thể viết bằng gần như tất cả các ngôn ngữ
  \\ \hline
  ~
  & • Push commands cho phép bạn kích hoạt thay đổi ngay lập tức
  & • Tập trung JSON dựa trên "túi dữ liệu" cho phép các kịch bản thu thập các biên trong thời gian chạy
  & • Không cần agent để có thể quản lý các client
  \\
  ~
  & • Giao diện web cho phép xử lý báo cáo, kiểm kê và quản lý nút theo thời gian thực
  & • Giao diện web cho phép bạn tìm kiếm các nút và inventory, xem hành động của nút, và gán Cookbooks, roles, nodes
  & • Giao diện web cho phép bạn cấu hình người dùng, các nhóm, và các thông tin cần thu thập, và áp dụng Playbooks tới các Inventory
  \\
  ~
  & • Chi tiết và báo cáo chuyên sâu về tình trạng của agnet và cấu hình của nút đó
  & ~
  & • Cực kì đơn giản để cài đặt và sử dụng
  \\ \hline
  \textbf{Nhược điểm}
  & • Yêu cầu phải học Puppet DSL hoặc Ruby
  & • Yêu cầu phải có sự hiểu biết nhất định về lập trình Ruby
  & • Không hỗ trợ các máy chủ Windows
  \\
  ~
  & • Quá trình cài đặt thiếu việc kiểm tra lỗi và báo cáo lỗi
  & • Hiện tại vẫn thiếu chức năng push command
  & • Giao diện người dùng web không tự động triển khai cùng Ansible; Các inventory phải nhập vào thủ công
  \\
  ~
  & ~
  & • Tài liệu đôi khi còn mơ hồ
  & ~
  \\ \hline
  \textbf{Chi phí}
  & • Phiên bản miễn phí mã nguồn mở
  & • Phiên bản miễn phí mã nguồn mở
  & • Phiên bản miễn phí mã nguồn mở
  \\
  ~
  & • Phiên bản dành cho doanh nghiệp Puppet Enterprise có giá 100\$/năm cho mỗi máy
  & • Phiên bản dành cho doanh nghiệp Enterprise Chef: miễn phí cho 5 máy, 120\$/tháng cho 20 máy, 300\$/tháng cho 50 máy và 600\$/tháng cho 100 máy
  & • Phiên bản dành cho doanh nghiệp AWX miễn phí cho 10 máy, 100\$ hoặ 250\$ một máy một năm tùy theo dịch vụ hỗ trợ
  \\ \hline
\end{longtable}
\end{landscape}

\pagestyle{fancy}